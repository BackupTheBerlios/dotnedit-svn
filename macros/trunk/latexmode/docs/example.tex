\documentclass[11pt]{article}

\usepackage{amsmath,amssymb,amsthm} 

\newcommand{\norm}[1]{\lVert#1\rVert}
\newcommand{\nedit}{N\kern-0.8pt{Edit}}

\begin{document} 

\section{Mathematics}
\noindent If things should go wrong notice that there is a unlimited undo command. So \textsc{strg+z} will help then.\\

Let us begin with checking the expander. Hit \textsc{Alt+u}, give in a dollar sign. (Then two dollars are put at a newline with the cursor in between. If you want to give in only a single dollar sign, hit  \textsc{Alt+4}. If you don't like this behavior, change the key bindings, see the X resource file.) Now give in \verb|'e| between the dollar signs and hit the \textsc{space} key. If it works, you get this:

$\epsilon$ 

Give in a comma directly behind the 
$\epsilon$ and hit two times \textsc{space}. Notice that the cursor walks throught the last dollar sign and  the comma moves behind the dollar sign. If this doesn't work, turn `Smart Indent' on. another test. As you see here `another' is not capitalized. Move the cursor directly behind the `r' and hit \textsc{space}, then `another' will be capitalized. If for abbreviations you don't want this to happen, you hit \textsc{shift+space}. Try it.\\

Now we come to somewhat bigger mathematical formulas.
Let us start with a small matrix.
This is a small matrix
$a b c
d e f$. Select from ``a'' to ``f'' and click the right mouse button. Now choose ``Matrices'', select the type of brackets and choose ``small''.\\


% These is an example from the amsldoc.tex, i.e. the AmS-LaTeX manual
Now a first equation 

% 1a
% A single split (i.e. spanning over several lines) equation. Notice that \begin-\end blocks like the matrix are seen as one line!
a=\begin{pmatrix}
x & y\\
y & x
\end{pmatrix}
=b+c-d
\quad +e-f
=g+h
=i

Select from ``a'' to the end, right mouse button, choose ``Equations'' and then ``single equation''.\\

% 1b
% Override the usual alignment with a ; 
a=b;=c-d
\quad +e-f
=g+h
=i


% 2a
% two gathered equations, separate with extra newline!
a_1=b_1+c_1

a_2=b_2+c_2-d_2+e_2

Select everything from \verb>a_1> to \verb<e_2<, and choose ``gathered equations''.
% And looking at the \verb stuff you are perhaps starting to wonder about the syntax highlighting of this little editor here. Just try any legal characters other than > or <.

% 2b
% three gathered equations, one of them over several lines
a_1=b_1+c_1

a=b+c-d
\quad +e-f
=g+h
=i

a_2=b_2+c_2-d_2+e_2

Another example of gathered equations one of which spans over several lines. Select the whole stuff.\\

Now to aligned equations:

% 3a
% two aligned equations
a_1=b_1+c_1
a_2=b_2+c_2-d_2+e_2


% 3b
% aligned columns of equations 
a_{11}=b_{11} ; a_{12}\leq b_{12}
a_{21}=b_{21} ; a_{22}=b_{22}+c_{22}

You separate the columns with \verb*} ; }, i.e. space, semicolon and another space.

% 3c
% columns with text
x= y_1-y_2+y_3-y_5+y_8-\dots ;; \text{by equation x}
= y'\circ y^* ;; \text{by euation y}
= y(0) y' ;; \text{by Axiom 1.}


% 4
% the aligned environment -> equations} describing text
As last example Maxwell's equations

B'=-\partial\times E,
E'=\partial\times B - 4\pi j,


Select them as usual and from the ``Equations'' menu choose ``align \}''. Then insert ``Maxwell's equations''.\\

\section{Completions}\label{comp}
After this short intro% 
to editing mathematics, here come a few other things. First completions. Put the cursor after `intro' before the procent and hit \textsc{f4} two times. Then you'll get `introduction'. To see a list of all possible completions, hit \textsc{f5}. You can select a completion by hitting a number. Now to code completions:
$\var$

Now we come to editing multi-file documents. Invoke `Sectioning' from the macro menu or hit \textsc{ctrl+alt+s}. Since you haven't set a main file so far, it is assumed that the current file is the main file. You can see the sectioning in a dialog box, select an entry (with mouse or keyboard) and jump to it.

To set a main file, invoke `Main File' from the macro menu (you can define a short-cut for it, too). You see a list of currently opened file and of files that you have bookmarked. To bookmark the current file, hit \textsc{ctrl+b} or got to `Bookmarks' in the macro menu. Hit \textsc{shift+ctrl+b} to see a list of your bookmarked files.

Don't forget to test also the `Labels' and `Command' macros.\\

When you test inverse search, notice that you \emph{must} run \nedit\ through the \verb|nc| command or the \verb|-server| option. Otherwise a new window will be opened, even if the file in question is already open!

\section{Multifile Documents}

Now we checking out editing multifile documents

\subsection{Another input file}

\section{This is file example2}
See if the sectioning and the labels work correctly. Check out also the bookmark feature (\textsc{Ctrl+B} and \textsc{shift+ctrl+b}).\label{in-example2}

\section{The End}
This should get you going. But to remind you again: These macros are by no means natural laws. You can change everything to your needs and taste. Just do it.


This above is to illustrate editing multifile documents.\label{example}


\section{Lists, Enumerations}

% Nesting list environments 
You can mix the list environments to your taste: But it might start to look silly.
With a dash.
Therefore remember: Stupid things will not become smart because they are
in a list.
Smart things, though, can be presented beautifully in a list.
% Select from `But it ...' to `With a dash.'. Right click, select Lists and then 
% `o o o' for itemize. Then select from `Stupid things...' to the end `..in a list.'. Again right click, Lists, and now `Descriptions'. Notice that `Stupid' and `Smart' are now in brackets. Finally select from start `You can mix...' to the end, ie \end{description}. Again Lists, now `1.2.3.' for enumeration.


% Another example
A small matrix $a b
c d$ 
inside a list.

% Select from a to d to insert a small matrix. Then select from start to end, right click, choose List and `(i) (ii) (iii)'. Notice this needs extra packages.


\end{document}
